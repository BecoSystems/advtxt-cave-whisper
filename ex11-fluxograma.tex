%%%%%%%%%%%%%%%%%%%%%%%%%%%%%%%%%%%%%%%%%%%%%%%%%%%%%%%%%%%%%%%%%%%%%%%%%%%%%%%%%%%%%%%%%
% Criação de Fluxograma usando LaTeX
%
% Assunto: Fluxograma do jogo "Cave Whisper" - Jogo de aventura em texto
%          com múltiplas escolhas e finais diferentes
%
% Autores:
%     Pedro Victor B. A. Souza
%     Jamerson David A. Queiroz
%
% Coordenação:
%     Prof. Dr. Ruben Carlo Benante
%
% Data: 2024-04-25
%%%%%%%%%%%%%%%%%%%%%%%%%%%%%%%%%%%%%%%%%%%%%%%%%%%%%%%%%%%%%%%%%%%%%%%%%%%%%%%%%%%%%%%%


%%%%%%%%%%%%%%%%%%%%%%%%%%%%%%%%%%%%%%%%%%%%%%%%%%%%%%%%%%%%%%%%%%%%%%%%%%%%%%%%%%%%%%%%
% Para gerar o PDF use o comando make com o makefile configurado:
%
%    $ make ex11-programa1-benante-souza-queiroz.pdf
%
% O conteúdo do makefile é composto dos 3 seguintes comandos que ficam assim automatizados:
%    $ pdflatex ex11-fluxograma.tex -o ex11-fluxograma.pdf
%    $ bibtex biblio
%    $ pdflatex ex11-fluxograma.tex -o ex11-fluxograma.pdf


%%%%%%%%%%%%%%%%%%%%%%%%%%%%%%%%%%%%%%%%%%%%%%%%%%%%%%%%%%%%%%%%%%%%%%%%%%%%%%%%%%%%%%%%
% preambulo %%%%%%%%%%%%%%%%%%%%%%%%%%%%%%%%%%%%%%%%%%%%%%%%%%%%%%%%%%%%%%%%%%%%%%%%%%%%
\documentclass[a4paper,12pt]{article} %twocolumn
\usepackage[left=2.5cm,right=2cm,top=2.5cm,bottom=2cm]{geometry}
\usepackage[utf8]{inputenc} % letras acentuadas
\usepackage[portuguese]{babel} % tradução de títulos
\usepackage[colorlinks]{hyperref}
\usepackage{tikz} % para adicionar fluxogramas
\usepackage{algorithm} % ambiente para índice de algoritmos
\usepackage{algpseudocode} % fonte e estilo do algoritmo
\usepackage{graphicx} % permite adicionar imagens
\usepackage{indentfirst} % indenta o primeiro parágrafo também
\usepackage{url} % permite adicionar links de URLs e emails

\DeclareUrlCommand\email{\urlstyle{mm}} % comando para email bonito
\floatname{algorithm}{Algoritmo} % tradução da palavra algoritimo no ambiente de índice

\usetikzlibrary{shapes.geometric, shapes.symbols,arrows} % ajuste do tikz para incluir formas e setas

%%%%%%%%%%%%%%%%%%%%%%%%%%%%%%%%%%%%%%%%%%%%%%%%%%%%%%%%%%%%%%%%%%%%%%%%%%%%%%%%%%%%%%%%
% capa %%%%%%%%%%%%%%%%%%%%%%%%%%%%%%%%%%%%%%%%%%%%%%%%%%%%%%%%%%%%%%%%%%%%%%%%%%%%%%%%%
\title{Fluxograma: ex11 - Cave Whisper}
\author{Pedro Victor B. A. Souza \\ Jamerson David A. Queiroz}

\begin{document}

\maketitle

%%%%%%%%%%%%%%%%%%%%%%%%%%%%%%%%%%%%%%%%%%%%%%%%%%%%%%%%%%%%%%%%%%%%%%%%%%%%%%%%%%%%%%%%
% definicao dos blocos do fluxograma (tikz) %%%%%%%%%%%%%%%%%%%%%%%%%%%%%%%%%%%%%%%%%%%%

\tikzstyle{line} = [draw, -latex']
\tikzstyle{startend} = [draw, ellipse,fill=red!20, minimum height=2em, node distance=1.55cm]
\tikzstyle{print} = [tape, fill=blue!20, draw, draw=black, minimum width=3cm, minimum height=1.4cm, text width=4.5em, text centered, tape bend top=none, tape bend height=0.2cm, node distance=1.55cm]
\tikzstyle{input} = [trapezium, trapezium left angle=60, trapezium right angle=90, minimum width=3cm, minimum height=1cm, text centered, draw=black, fill=blue!30, node distance=1.95cm]
\tikzstyle{process} = [rectangle, minimum width=3cm, minimum height=1cm, text centered, draw=black, fill=orange!30, node distance=1.55cm]
\tikzstyle{decision} = [diamond, minimum width=3cm, minimum height=1cm, text centered, draw=black, fill=green!30, node distance=2.25cm]

%%%%%%%%%%%%%%%%%%%%%%%%%%%%%%%%%%%%%%%%%%%%%%%%%%%%%%%%%%%%%%%%%%%%%%%%%%%%%%%%%%%%%%%%
% resumo %%%%%%%%%%%%%%%%%%%%%%%%%%%%%%%%%%%%%%%%%%%%%%%%%%%%%%%%%%%%%%%%%%%%%%%%%%%%%%%

\begin{abstract}

\textbf{Assunto:} Fluxograma do jogo "Cave Whisper"

Jogo de aventura em texto com múltiplas escolhas onde o jogador encontra uma caverna misteriosa e deve tomar decisões que levam a diferentes finais. O jogo apresenta três objetos principais (tocha, espelho e livro antigo) e múltiplos caminhos possíveis.

Após a modelagem do fluxograma e desenvolvimento da lógica de programação em algoritmo, o programa será implementado na Linguagem de Programação \texttt{C}.

\textbf{Local:} Escola Politécnica de Pernambuco - UPE/POLI

\textbf{Órgão Financiador:} N/A

\textbf{Caracterização:} Modelagem, Projeto e Implementação de Software em Linguagem \texttt{C}

\end{abstract}

%%%%%%%%%%%%%%%%%%%%%%%%%%%%%%%%%%%%%%%%%%%%%%%%%%%%%%%%%%%%%%%%%%%%%%%%%%%%%%%%%%%%%%%%
% artigo %%%%%%%%%%%%%%%%%%%%%%%%%%%%%%%%%%%%%%%%%%%%%%%%%%%%%%%%%%%%%%%%%%%%%%%%%%%%%%%
% seção de introdução %%%%%%%%%%%%%%%%%%%%%%%%%%%%%%%%%%%%%%%%%%%%%%%%%%%%%%%%%%%%%%%%%%
\section{Introdução}

O jogo "Cave Whisper" é uma aventura interativa em texto onde o jogador se encontra em uma floresta escura e descobre uma caverna misteriosa. O jogo apresenta múltiplas escolhas que levam a diferentes finais, alguns positivos e outros negativos.

O programa foi modelado em \textit{fluxograma} para representar visualmente todas as possíveis decisões e seus resultados. Em seguida, a lógica foi desenvolvida em formato de \textit{algoritmo} e finalmente implementada em Linguagem de Programação \texttt{C}.

%%%%%%%%%%%%%%%%%%%%%%%%%%%%%%%%%%%%%%%%%%%%%%%%%%%%%%%%%%%%%%%%%%%%%%%%%%%%%%%%%%%%%%%%
% seção de objetivos %%%%%%%%%%%%%%%%%%%%%%%%%%%%%%%%%%%%%%%%%%%%%%%%%%%%%%%%%%%%%%%%%%%
\section{Fluxograma}

\begin{tikzpicture}[node distance=1.5cm]
    % Nodos principais
    \node (inicio) [startend] {Início};
    \node (intro) [print, below of=inicio] {Bem-vindo ao Cave Whisper};
    \node (cenario) [print, below of=intro] {Você está numa floresta escura e avista uma caverna};
    \node (bau) [decision, below of=cenario] {Abrir o baú?};
    
    % Ramificação não abre baú
    \node (naoabre) [print, right of=bau, xshift=3cm] {Não abre};
    \node (fim1) [startend, below of=naoabre] {Fim: Você morre};
    
    % Ramificação abre baú
    \node (abre) [print, below of=bau] {Abre o baú};
    \node (objetos) [decision, below of=abre] {Escolher objeto};
    
    % Objetos
    \node (tocha) [print, left of=objetos, xshift=-3cm, yshift=-1cm] {Tocha};
    \node (espelho) [print, below of=objetos] {Espelho};
    \node (livro) [print, right of=objetos, xshift=3cm, yshift=-1cm] {Livro};
    
    % Ações para cada objeto
    \node (acao_tocha) [decision, below of=tocha] {Ação};
    \node (acao_espelho) [decision, below of=espelho] {Ação};
    \node (acao_livro) [decision, below of=livro] {Ação};
    
    % Finais
    \node (fim_tocha1) [startend, left of=acao_tocha, xshift=-2cm] {Morre};
    \node (fim_tocha2) [startend, below of=acao_tocha] {Morre};
    \node (fim_tocha3) [startend, right of=acao_tocha, xshift=2cm] {Sobrevive};
    
    \node (fim_espelho1) [startend, left of=acao_espelho, xshift=-2cm] {Morre};
    \node (fim_espelho2) [startend, below of=acao_espelho] {Morre};
    \node (fim_espelho3) [startend, right of=acao_espelho, xshift=2cm] {Sobrevive};
    
    \node (fim_livro1) [startend, left of=acao_livro, xshift=-2cm] {Sobrevive};
    \node (fim_livro2) [startend, below of=acao_livro] {Sobrevive};
    \node (fim_livro3) [startend, right of=acao_livro, xshift=2cm] {Sobrevive};
    
    % Conexões
    \path [line] (inicio) -- (intro);
    \path [line] (intro) -- (cenario);
    \path [line] (cenario) -- (bau);
    
    % Ramificação principal
    \path [line] (bau) -- node[above] {Não} (naoabre);
    \path [line] (naoabre) -- (fim1);
    \path [line] (bau) -- node[right] {Sim} (abre);
    \path [line] (abre) -- (objetos);
    
    % Objetos
    \path [line] (objetos) -| node[above] {Tocha} (tocha);
    \path [line] (objetos) -- node[right] {Espelho} (espelho);
    \path [line] (objetos) -| node[above] {Livro} (livro);
    
    % Ações tocha
    \path [line] (tocha) -- (acao_tocha);
    \path [line] (acao_tocha) -- node[above] {Entrar} (fim_tocha1);
    \path [line] (acao_tocha) -- node[right] {Esperar} (fim_tocha2);
    \path [line] (acao_tocha) -- node[above] {Fugir} (fim_tocha3);
    
    % Ações espelho
    \path [line] (espelho) -- (acao_espelho);
    \path [line] (acao_espelho) -- node[above] {Entrar} (fim_espelho1);
    \path [line] (acao_espelho) -- node[right] {Esperar} (fim_espelho2);
    \path [line] (acao_espelho) -- node[above] {Fugir} (fim_espelho3);
    
    % Ações livro
    \path [line] (livro) -- (acao_livro);
    \path [line] (acao_livro) -- node[above] {Entrar} (fim_livro1);
    \path [line] (acao_livro) -- node[right] {Esperar} (fim_livro2);
    \path [line] (acao_livro) -- node[above] {Fugir} (fim_livro3);
\end{tikzpicture}

\clearpage

%%%%%%%%%%%%%%%%%%%%%%%%%%%%%%%%%%%%%%%%%%%%%%%%%%%%%%%%%%%%%%%%%%%%%%%%%%%%%%%%%%%%%%%%
% Autores %%%%%%%%%%%%%%%%%%%%%%%%%%%%%%%%%%%%%%%%%%%%%%%%%%%%%%%%%%%%%%%%%%%%%%%%%%%%%%
\section*{Detalhamento dos Autores}

\subsection*{Discentes}

\begin{enumerate}
    \item \textbf{Nome Completo:} Pedro Victor B. A. Souza
    \begin{description}
        \item [Email:] \email{pvbas@poli.br}
        \item [Endereço:]
        \item [Matrícula:]
        \item [CPF:]
        \item [RG:]
        \item [Telefone:]
        \item [Currículo Lattes:]
    \end{description}

    \item \textbf{Nome Completo:} Jamerson David A. Queiroz
    \begin{description}
        \item [Email:] \email{jdaq@poli.br}
        \item [Endereço:]
        \item [Matrícula:]
        \item [CPF:]
        \item [RG:]
        \item [Telefone:]
        \item [Currículo Lattes:]
    \end{description}
\end{enumerate}

\subsection*{Docentes}

\begin{enumerate}
    \item \textbf{Nome Completo:} Ruben Carlo Benante
    \begin{description}
        \item [Email:] \email{rcb@upe.br}
        \item [Matrícula:] 11238-0
        \item [Currículo Lattes:] \url{http://lattes.cnpq.br/3366717378277623}
    \end{description}
\end{enumerate}

%%%%%%%%%%%%%%%%%%%%%%%%%%%%%%%%%%%%%%%%%%%%%%%%%%%%%%%%%%%%%%%%%%%%%%%%%%%%%%%%%%%%%%%%
% referências bibliográficas %%%%%%%%%%%%%%%%%%%%%%%%%%%%%%%%%%%%%%%%%%%%%%%%%%%%%%%%%%%
\nocite{*}
\bibliographystyle{acm}
\bibliography{biblio}

\end{document}
